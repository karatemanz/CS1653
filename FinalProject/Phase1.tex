\documentclass[11pt]{article}
\usepackage{amsmath}
\usepackage{geometry}                % See geometry.pdf to learn the layout options. There are lots.
\geometry{letterpaper}                   % ... or a4paper or a5paper or ... 
%\geometry{landscape}                % Activate for for rotated page geometry
%\usepackage[parfill]{parskip}    % Activate to begin paragraphs with an empty line rather than an indent
\usepackage{graphicx}
\usepackage{amssymb}
\usepackage{epstopdf}
\DeclareGraphicsRule{.tif}{png}{.png}{`convert #1 `dirname #1`/`basename #1 .tif`.png}

%Don't list section numbers
\setcounter{secnumdepth}{0}

\title{CS 1653: Applied Cryptography and Network Security\\Term Project, Phase 1}
\author{Lindsey Bieda\quad\texttt{leb35@pitt.edu}\qquad Tucker Trainor\quad\texttt{tmt33@pitt.edu}}
\date{January 19, 2012} % Activate to display a given date or no date

\begin{document}
\maketitle
\section{Section 1: Group Information}
Lindsey Bieda\qquad\texttt{leb35@pitt.edu}
\newline
Tucker Trainor\qquad\texttt{tmt33@pitt.edu}
\section{Section 2: Security Requirements}
\subsection{System Assumptions \& Threat Models}
% A paragraph informally describing the trust assumptions that you are making regarding the players in the system. This could, for example, be a description of the system model in which you envision your application being deployed.
% Threat model 1

Preventing malicious users from accessing the system or authorized users from defacing existing data.
\begin{enumerate}
\item \textbf{Property 1: User verification.}
\item \textbf{Property 2: User/group uniqueness.}
\item \textbf{Property 3: Strong login protocols.}
\item \textbf{Property 4: Connection restrictions.}
\item \textbf{Property 5: Password integrity (and authentication).}
\item \textbf{Property 6: Filename uniqueness.}
\item \textbf{Property 7: Physical system protections.}
\end{enumerate}
% Threat model 2
Privacy and isolation of files must be ensured if users are to trust the file sharing system.
\begin{enumerate}
\item \textbf{Property 1: Correctness.} Correctness implies that if file \emph{f} is shared with members of group \emph{g}, only members of group {g} should be able to access \emph{f}. The notion of ``access� entails the creation, modification, and deletion of \emph{f}, as well as the ability to see that \emph{f} even exists. Without this requirement, any user could access any file, which is contrary to the notion of group-based file sharing.
\item \textbf{Property 2: Permission levels.}
\item \textbf{Property 3: File integrity.}
\item \textbf{Property 4: File restrictions.}
\item \textbf{Property 5: User roles.}
\item \textbf{Property 6: download integrity.}
\item \textbf{Property 7: Open verified software.} Open verified software implies that the software installed on the system will be open source and heavily validated and supported by the community in order to ensure both correctness and security. Open verified software provides transparency in source code and by extension software, allowing enhanced security over black box software.
\end{enumerate}
\end{document}
